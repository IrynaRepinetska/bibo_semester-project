\documentclass[9pt]{beamer}

\usepackage{amsmath,amsthm,amssymb} %fuer math. Formeln
\usepackage{graphicx} %fuer Bilder
\usepackage{color} %fuer farbigen Text
\usepackage{listings}
\lstset{
  language=XML,
	backgroundcolor = \color{lightgray},
  basicstyle=\tiny,
columns=fullflexible
}

\usepackage[utf8]{inputenc}
\DeclareUnicodeCharacter{00A0}{ }

\usetheme{Berlin}  % Vorlage, welche das Aussehen der Praesentation festlegt-andere Alternativen: AnnArbor, Bergen,...
\usecolortheme{orchid} % Vorlage, welche das Farblayout der Praesentation festlegt -andere Alternativen: lily, beetle,...
\author{Iryna Repinetska, Chris Roeseler}
\title{AIML - Artificial Intelligence Markup Language}
\institute{Institut f\"ur Informatik Humboldt-Universit\"at zu Berlin}
\date{\today} 



\begin{document}

\begin{frame}% Beginne eine Slide 
  \titlepage % Inhalt (In diesem Fall die Titelseite)
\end{frame} % Beende eine Slide

\section{Einleitung}
\begin{frame}
\begin{itemize}
\item Die kognitiven Schnittstellen als die neue Form des Zusammenspiels zwischen Menschen und Maschienen.
\item Ein Beispiel von solchen kognitiven Oberflächen sind Chatbots oder Programme, die ein Dialog mit Menschen simulieren sollen.
\item Artificial Intelligence Markup Language (AIML) wurde in den Jahren von 1995 bis 2000  by Richard Wallace entwickelt und ist heuzutage am meisten benutzte Sprache für die Entwicklung von Chatbots.
\end{itemize}

\end{frame}
\section{Artificial Intelligence Markup Language}
\subsection{Allgemeines}
\begin{frame}
\begin{itemize}
\item Artificial Intelligence Markup Language basiert auf den Konzepten der Mustererkennung.
\item  AIML ist eine XML-basierte sowie tag-basierte Markup Sprache.
\item Die Allgemeine Form eines AIML Befehls hat die folgede Struktur:
$$<Befehl> ParameterListe </Befehl>$$.
\item Die grundlegende Einheiten des Dialoges werden Kategorien (categories) gennant und bilden die Wissensbasis des Chatbots.
\item Das AIML-Vokabular besteht aus Wörtern, Leerzeichen und den Sonderzeichen $\**$ und $\_$ .
\end{itemize}



\end{frame}



\section{Tags} 
\subsection{Basic Tags}
\begin{frame}[fragile]
  \frametitle{\textless aiml\textgreater}
\begin{itemize}
  \item Markiert Anfang und Ende des AIML Dokuments\\
  \item Optional Angabe von Version und Kodierung\\
  \item Vergleichbar mit \textless html\textgreater\textless /html\textgreater\\
\end{itemize}
  \begin{lstlisting}
 <aiml version="1.0.1" encoding="UTF-8"?> 
 <category> 
    <pattern> HELLO </pattern> 
    <template>  
       Hello User
    </template> 
 </category> 
 </aiml> 
  \end{lstlisting}
\end{frame}

\begin{frame}[fragile]
  \frametitle{\textless category\textgreater}  % Ueberschrift einer Seite
\begin{itemize}
  \item Genannt knowledge unit\\
  \item Enthält:\ \
\begin{itemize}
    \item Möglichen Anfrage String(\textless pattern\textgreater)\\
    \item Antwort/en des Bots(\textless template\textgreater)\\
    \item Optional: Kontext(\textless topic\textgreater)\\
  \end{itemize}
  \end{itemize}
  \begin{lstlisting}
 <aiml version="1.0.1" encoding="UTF-8"?> 
 <category> 
    <pattern> HELLO </pattern> 
    <template> 
       Hello User 
    </template> 
 </category> 
 </aiml> 
  \end{lstlisting}
\end{frame}

\begin{frame}[fragile]
  \frametitle{\textless pattern\textgreater}  % Ueberschrift einer Seite
\begin{itemize}
  \item Matched den user input\\
  \item Kann wildcard character enthalten\\
  \item case insensitiv\\
  \end{itemize}
  \begin{lstlisting}
 <aiml version="1.0.1" encoding="UTF-8"?> 
 <category> 
    <pattern> HELLO </pattern> 
    <template>  
       Hello User
    </template> 
 </category> 
 </aiml> 
  \end{lstlisting}
\end{frame}


\begin{frame}[fragile]
  \frametitle{\textless template\textgreater}
\begin{itemize}
  \item Antwort des Bots\\
  \item Informationen speichern für spätere Gespräche\\
  \item Programme aufrufen\\
  \item Weiterführende Fragen stellen\\
  \end{itemize}
  \begin{lstlisting}
 <aiml version="1.0.1" encoding="UTF-8"?> 
 <category> 
    <pattern> HELLO </pattern> 
    <template>  
       Hello User
    </template> 
 </category> 
 </aiml> 
  \end{lstlisting}
\end{frame}

\begin{frame}[fragile]
  \frametitle{\textless star\textgreater}
\begin{itemize}
  \item Korrespondiert mit wildcard character\\
  \item Indizierung aus pattern möglich
  \end{itemize}

  \begin{lstlisting}
  <category>
    <pattern> I AM * FROM * </pattern>

    <template>
      <star index=``2''/> is a nice region... <star index=``1''/> is a shit name though.
    </template>

  </category>
  \end{lstlisting}
\end{frame}

\begin{frame}[fragile]
  \frametitle{\textless srai\textgreater}
\begin{itemize}
\item Matched ein \textless pattern\textgreater und stellt die Anfrage geändert neu

\item Divide and Conquer - Beispiel Bot

\item Symbolic Reduction - Pattern vereinfachen

\item Synonyms resolution

\item Keywords detection
\end{itemize}
  \begin{lstlisting}
 <category> 
    <pattern> HELLO </pattern> 
    <template>  
       Hello User
    </template> 
 </category> 
  \end{lstlisting}
  \begin{lstlisting}
   <category>
     <pattern> HELLO *</pattern>

     <template>
       <srai>HELLO</srai>
     </template>
   </category>
  \end{lstlisting}
\end{frame}

\begin{frame}[fragile]
  \frametitle{\textless random\textgreater}
\begin{itemize}
\item verschiedene Antwortmöglichkeiten die zufällig ausgegeben werden\\
  \end{itemize}
\begin{lstlisting}

   <category>
      <pattern>HELLO</pattern>
      <template>
         <random>
            <li> Hello! </li>
            <li> Hi! Nice to meet you! </li>
         </random>
      </template>
   <category>

\end{lstlisting}
\end{frame}

\begin{frame}[fragile]
  \frametitle{\textless that\textgreater}
\begin{itemize}
  \item Kontext Antworten\\
  \item \textless that\textgreater : zuletzt ausgegebenes template\\
  \item Zusätzliche notwendige Bedingung beim matching\\
  \end{itemize}
\begin{lstlisting}
   <category>
      <pattern>WHAT ABOUT MOVIES</pattern>
      <template>Do you like comedy movies</template>  
   </category>
   
   <category>
      <pattern>YES</pattern>
      <that>Do you like comedy movies</that>
      <template>Nice, I like comedy movies too.</template>
   </category>
   
   <category>
      <pattern>NO</pattern>
      <that>Do you like comedy movies</that>
      <template>Ok! But I like comedy movies.</template>
   </category> 
\end{lstlisting}
\end{frame}

\begin{frame}[fragile]
  \frametitle{\textless set\textgreater\textless get\textgreater\textless think\textgreater}
\begin{itemize}
  \item Speichern von Variablen z.B. Name\\
  \end{itemize}
  \begin{lstlisting}
   <category>
      <pattern>I am *</pattern>
      <template>
         Hello <set name = "username"> <star/> </set>!
      </template>  
   </category>  
   
   <category>
      <pattern>Good Night</pattern>
      <template>
         Bye<get name = "username"/> Thanks for the conversation!
      </template>  
   </category>  
\end{lstlisting}
\begin{itemize}
\item think speichert im Hintergrund
\end{itemize}
  \begin{lstlisting}
      <template>
         Hello <think><set name = "username"> <star/>! </set></think>
      </template>
\end{lstlisting}

\end{frame}

\begin{frame}[fragile]
  \frametitle{\textless topic\textgreater}
\begin{itemize}
  \item Definiert catergory's die nur bei gesetztem topic matchen
\end{itemize}
\begin{lstlisting}
   <category>
      <pattern>LET DISCUSS MOVIES</pattern>
      <template>Yes <set name = "topic">movies</set></template>  
   </category>
   
   <topic name = "movies">
      <category>
         <pattern> * </pattern>
         <template>Watching good movie refreshes our minds.</template>
      </category>
      
      <category>
         <pattern> I LIKE WATCHING COMEDY! </pattern>
         <template>I like comedy movies too.</template>
      </category>
      
      <category>
         <pattern> STOP TALKING ABOUT MOVIES! </pattern>
         <template>OK.<think><set name = ``topic''></set></think></template>
      </category>
   </topic>
\end{lstlisting}
\end{frame}

\begin{frame}[fragile]
  \frametitle{\textless condition\textgreater}
\begin{itemize}
\item Ähnlich zu break
\item Bedingte Antworten im template
    \end{itemize}
      \begin{lstlisting}
      <category>  
        <pattern> HOW ARE YOU? </pattern>  
        <template>  
         <random>                                                                                           
           <li><think><set name = ``state''> happy</set></think></li>                           
           <li><think><set name = ``state''> sad  </set></think></li>                             
         </random>
       <condition name="state" value="happy">
         I am happy.
       </condition>
       <condition name="state" value="sad">
         I am sad.
       </condition>
    </template>
  </category>
	\end{lstlisting}
\end{frame}
\section{Zusammenfassung}
\begin{frame}


AIML ist heuzutage die am weit verbreitetse Programmiersprache für die Entwicklung von Chatbots. Diese Verbreitung beruht auf folgenden Gründen:
\begin{itemize}
\item AIML ist durch eine einfache Anwendung gekennzeichnet. Da sie auf XML (eXtensible Markup Language) basiert und die Implentation von Dialogen unter Verwendung von Tags einfacher ist;
\item es gibt verschiedene AIML-Editoren sowie Entwicklungsplattformen, die den Chatbot-Entwicklern bei der Codeerstellung sowie Web-Bereitstellung des Chatbots helfen;
\item Der wesentliche Teil aller Chatbot-Projekte, die mit AIML implementiert wurden, sind Open Source Software. Das gibt die Möglichkeit Source Code sowie die entsprechende Dokumentation für die neuen Projekte zu benutzen.
\end{itemize}

\end{frame}


\end{document}
